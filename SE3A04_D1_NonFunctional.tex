\newcommand{\nonfunctional}{
\section{Non-Functional Requirements}
\label{sec:non-functional_requirements}

% Begin Section
\subsection{Look and Feel Requirements}
\label{sub:look_and_feel_requirements}
% Begin SubSection

\subsubsection{Appearance Requirements}
\label{ssub:appearance_requirements}
% Begin SubSubSection
\begin{enumerate}[{LF-A}1. ]
	\item The operator dashboard shall display environmental data using visually distinct components (charts, maps, indicators) that clearly separate critical alerts from informational data. \\
	{\bf Rationale:} This ensures that operators can quickly identify and respond to critical alerts without confusion, improving response times during environmental events.
	\item Alert states shall be visually emphasized using multiple cues (color, iconography, labels).\\
	{\bf Rationale:} This enhances the visibility of critical information, ensuring that operators can easily distinguish between different alert levels and take appropriate action.
	\item The system shall provide visual cues for historical trends, such as shaded areas for previous hour/day metrics on graphs.\\
	{\bf Rationale:} This allows operators to contextualize current readings against recent trends, aiding in situational awareness and decision-making [6].
\end{enumerate}
% End SubSubSection

\subsubsection{Style Requirements}
\label{ssub:style_requirements}
% Begin SubSubSection
\begin{enumerate}[{LF-S}1. ]
	\item The dashboard interface shall prioritize clarity and information density over decorative visual elements.\\
	{\bf Rationale:} Environmental monitoring systems are operational tools. Minimalist interfaces improve scanning speed and reduce distraction [3].
	\item Icons used in the interface shall follow widely recognized conventions.\\
	{\bf Rationale:} Standardized symbols reduce training requirements and minimize interpretation\\ errors [4].
	\item The dashboard shall employ consistent font families and sizes across all modules to maintain readability and visual cohesion.\\
	{\bf Rationale:} Consistent typography supports faster scanning, reduces cognitive load, and improves professional appearance.
\end{enumerate}
% End SubSubSection

% End SubSection

\subsection{Usability and Humanity Requirements}
\label{sub:usability_and_humanity_requirements}
% Begin SubSection

\subsubsection{Ease of Use Requirements}
\label{ssub:ease_of_use_requirements}
% Begin SubSubSection
\begin{enumerate}[{UH-EOU}1. ]
	\item A trained operator shall be able to acknowledge an active alert within 30 seconds of dashboard interaction.\\
	{\bf Rationale:} Rapid acknowledgement is essential for incident management workflows and aligns with project usability goals.
	\item Core dashboard tasks (view alerts, inspect zone metrics, filter data) shall require no more than three interactions.\\
	{\bf Rationale:} Reducing interaction complexity improves efficiency and lowers error probability during time-sensitive events.
	\item System response time for standard dashboard actions shall not exceed 2 seconds under normal conditions.\\
	{\bf Rationale:} Perceived latency negatively affects usability and decision-making confidence [5].
\end{enumerate}
% End SubSubSection

\subsubsection{Personalization and Internationalization Requirements}
\label{ssub:personalization_and_internationalization_requirements}
% Begin SubSubSection
\begin{enumerate}[{UH-PI}1. ]
	\item The system shall support configurable measurement units without requiring code changes.\\
	{\bf Rationale:}Different municipalities and regulatory bodies use varying measurement standards.
	\item Textual content shall be externalized to enable future localization for multilingual support.\\
	{\bf Rationale:} The system may later require multilingual support. Designing for this extensibility avoids refactoring.
\end{enumerate}
% End SubSubSection

\subsubsection{Learning Requirements}
\label{ssub:learning_requirements}
% Begin SubSubSection
\begin{enumerate}[{UH-L}1. ]
	\item A new operator shall be able to understand basic dashboard functionality within one hour using provided documentation.\\
	{\bf Rationale:} Municipal environments frequently involve staff turnover and cross-training. The onboarding process shouldn't take too long, but since this is a specialized system, some training is expected.
	It also requires consideration for safety-critical operations, so operators must have a solid understanding of the system before using it in a live environment.
\end{enumerate}
% End SubSubSection

\subsubsection{Understandability and Politeness Requirements}
\label{ssub:understandability_and_politeness_requirements}
% Begin SubSubSection
\begin{enumerate}[{UH-UP}1. ]
	\item System notifications and alerts shall use concise, non-technical language.\\
	{\bf Rationale:} Operators are domain users, not necessarily software specialists.
\end{enumerate}
% End SubSubSection

\subsubsection{Accessibility Requirements}
\label{ssub:accessibility_requirements}
% Begin SubSubSection
\begin{enumerate}[{UH-A}1. ]
	\item Anything that is not text must also have a text alternative where applicable (ex. Alt text for images).\\
	{\bf Rationale:} This ensures that users with visual impairments can still access the content using screen readers [2].
	\item All user interface components must have a contrast ratio of at least 3:1. \\
	{\bf Rationale:} This improves readability for users with visual impairments and ensures compliance with WCAG 2.1 guidelines [2].
\end{enumerate}
% End SubSubSection

% End SubSection

\subsection{Performance Requirements}
\label{sub:performance_requirements}
% Begin SubSection

\subsubsection{Speed and Latency Requirements}
\label{ssub:speed_and_latency_requirements}
% Begin SubSubSection
\begin{enumerate}[{PR-SL}1. ]
	\item 
\end{enumerate}
% End SubSubSection

\subsubsection{Safety-Critical Requirements}
\label{ssub:safety_critical_requirements}
% Begin SubSubSection
\begin{enumerate}[{PR-SC}1. ]
	\item 
\end{enumerate}
% End SubSubSection

\subsubsection{Precision or Accuracy Requirements}
\label{ssub:precision_or_accuracy_requirements}
% Begin SubSubSection
\begin{enumerate}[{PR-PA}1. ]
	\item 
\end{enumerate}
% End SubSubSection

\subsubsection{Reliability and Availability Requirements}
\label{ssub:reliability_and_availability_requirements}
% Begin SubSubSection
\begin{enumerate}[{PR-RA}1. ]
	\item 
\end{enumerate}
% End SubSubSection

\subsubsection{Robustness or Fault-Tolerance Requirements}
\label{ssub:robustness_or_fault_tolerance_requirements}
% Begin SubSubSection
\begin{enumerate}[{PR-RFT}1. ]
	\item 
\end{enumerate}
% End SubSubSection

\subsubsection{Capacity Requirements}
\label{ssub:capacity_requirements}
% Begin SubSubSection
\begin{enumerate}[{PR-C}1. ]
	\item 
\end{enumerate}
% End SubSubSection

\subsubsection{Scalability or Extensibility Requirements}
\label{ssub:scalability_or_extensibility_requirements}
% Begin SubSubSection
\begin{enumerate}[{PR-SE}1. ]
	\item 
\end{enumerate}
% End SubSubSection

\subsubsection{Longevity Requirements}
\label{ssub:longevity_requirements}
% Begin SubSubSection
\begin{enumerate}[{PR-L}1. ]
	\item 
\end{enumerate}
% End SubSubSection

% End SubSection

\subsection{Operational and Environmental Requirements}
\label{sub:operational_and_environmental_requirements}
% Begin SubSection

\subsubsection{Expected Physical Environment}
\label{ssub:expected_physical_environment}
% Begin SubSubSection
\begin{enumerate}[{OE-EPE}1. ]
	\item 
\end{enumerate}
% End SubSubSection

\subsubsection{Requirements for Interfacing with Adjacent Systems}
\label{ssub:requirements_for_interfacing_with_adjacent_systems}
% Begin SubSubSection
\begin{enumerate}[{OE-IA}1. ]
	\item 
\end{enumerate}
% End SubSubSection

\subsubsection{Productization Requirements}
\label{ssub:productization_requirements}
% Begin SubSubSection
\begin{enumerate}[{OE-P}1. ]
	\item 
\end{enumerate}
% End SubSubSection

\subsubsection{Release Requirements}
\label{ssub:release_requirements}
% Begin SubSubSection
\begin{enumerate}[{OE-R}1. ]
	\item 
\end{enumerate}
% End SubSubSection

% End SubSection

\subsection{Maintainability and Support Requirements}
\label{sub:maintainability_and_support_requirements}
% Begin SubSection

\subsubsection{Maintenance Requirements}
\label{ssub:maintenance_requirements}
% Begin SubSubSection
\begin{enumerate}[{MS-M}1. ]
	\item The system must receive a minimum of bi-monthly software updates to fix bugs and improve performance. \\
	{\bf Rationale:} Regular software updates are important to ensure that any bugs are addressed and performance is improved over time. This also helps to maintain user trust and satisfaction with the system.
\end{enumerate}
% End SubSubSection

\subsubsection{Supportability Requirements}
\label{ssub:supportability_requirements}
% Begin SubSubSection
\begin{enumerate}[{MS-S}1. ]
	\item The system must provide users with access to help or support information within the application. \\
	{\bf Rationale:} Providing users with access to help or support information within the application can improve user experience and reduce frustration. It allows users to quickly find answers to their questions or troubleshoot issues without having to leave the application or contact support directly. This can also help to reduce the workload on support staff by giving users the ability to resolve issues on their own.
\end{enumerate}

\begin{enumerate}[{MS-S}2. ]
	\item The system must provide a way for users to contact support staff. \\
	{\bf Rationale:} If users cannot find a solution to their problem through the support information it is important that they have an easy way to contact support staff. This can help to ensure that users receive the assistance they need in a timely manner, which can improve user satisfaction.
\end{enumerate}
% End SubSubSection

\subsubsection{Adaptability Requirements}
\label{ssub:adaptability_requirements}
% Begin SubSubSection
\begin{enumerate}[{MS-A}1. ]
	\item The system must be able to operate on iOS 15 and above on all iOS devices. \\
	{\bf Rationale:} This system is being developed for iOS and it should be able to run on the more recent versions of iOS to ensure that it can be used by a wide range of users. By supporting iOS 15 and above, the system can take advantage of the latest features and improvements in the operating system, which can enhance user experience and performance.
\end{enumerate}
% End SubSubSection

% End SubSection

\subsection{Security Requirements}
\label{sub:security_requirements}
% Begin SubSection

\subsubsection{Access Requirements}
\label{ssub:access_requirements}
% Begin SubSubSection
\begin{enumerate}[{SR-AC}1. ]
	\item The system must require users to to authenticate themselves before they can access the system. \\
	{\bf Rationale:} This is important to ensure that only authorized users can access the system and its data. Authentication can help to prevent unauthorized access and protect sensitive information from being accessed.
\end{enumerate}

\begin{enumerate}[{SR-AC}2. ]
	\item The system shall enforce role-based access control for all users. \\
	{\bf Rationale:} Role-based access control is important to ensure that city operators can access more detailed information about alerts whereas general users can only access basic information. This helps to protect sensitive information and ensure that users only have access to the information that is relevant to their role.
\end{enumerate}

\begin{enumerate}[{SR-AC}3. ]
	\item The system shall restrict administrative functions to users assigned an operator role. \\
	{\bf Rationale:} Only city operators should have access to certain adminsitrative functiosn such as creating/updating alert rules, resolving alerts, etc. This helps to ensure that only authorized users can perform these actions and helps to protect the integrity of the system.
\end{enumerate}
% End SubSubSection

\subsubsection{Integrity Requirements}
\label{ssub:integrity_requirements}
% Begin SubSubSection
\begin{enumerate}[{SR-INT}1. ]
	\item The system shall validate all incoming telemetry data. \\
	{\bf Rationale:} Validating the data is important to ensure it conforms to the expected format and schema.
\end{enumerate}

\begin{enumerate}[{SR-INT}2. ]
	\item The system shall reject telemetry data that fails plausibility or validation checks. \\
	{\bf Rationale:} Rejecting invalid data is important to maintain the integrity of the system and prevent it from being compromised by corrupted data.
\end{enumerate}
% End SubSubSection

\subsubsection{Privacy Requirements}
\label{ssub:privacy_requirements}
% Begin SubSubSection
\begin{enumerate}[{SR-P}1. ]
	\item The system shall encrypt any sensitive personal information collected from users. \\
	{\bf Rationale:} Encrypting sensitive personal information is important to protect user privacy and prevent unauthorized access to this information. This can help to build trust with users and ensure that their data is handled securely.
\end{enumerate}

\begin{enumerate}[{SR-P}2. ]
	\item The system shall provide users with the ability to delete their personal information and account upon request. \\
	{\bf Rationale:} Providing users with the ability to delete their personal information and account is important to give users control over their data and respect their privacy. This ensures that users feel comfortable using the system.
\end{enumerate}

\begin{enumerate}[{SR-P}3. ]
	\item The system shall provide users with a privacy notice that informs them how their information is being used and stored. \\
	{\bf Rationale:} Providing users with a privacy notice is important to ensure transparency and build trust with users. It allows users to understand how their information is being used and stored, which can help to alleviate concerns about privacy and encourage them to use the system.
\end{enumerate}
% End SubSubSection

\subsubsection{Audit Requirements}
\label{ssub:audit_requirements}
% Begin SubSubSection
\begin{enumerate}[{SR-AU}1. ]
	\item The system must store a history of alerts including any changes made, the user who made the changes, and the timestamp. \\
	{\bf Rationale:} Storing a history of alerts and changes is important for maintaining a record of actions taken within the system.
\end{enumerate}

\begin{enumerate}[{SR-AU}2. ]
	\item The system shall log all authentication attempts, including successful and failed logins. \\
	{\bf Rationale:} Logging authentication attempts is important for security monitoring. It allows administrators to detect and investigate potential security breaches or unauthorized access attempts.
\end{enumerate}
% End SubSubSection

\subsubsection{Immunity Requirements}
\label{ssub:immunity_requirements}
% Begin SubSubSection
\begin{enumerate}[{SR-IM}1. ]
	\item The system shall limit repeated failed authentication attemps. \\
	{\bf Rationale:} After a certain number of failed login attempts, the system should temporarily lock the account or require additional verification to prevent attacks and protect user accounts from unauthorized access.
\end{enumerate}

\begin{enumerate}[{SR-IM}2. ]
	\item The system shall protect against unauthorized access through multi-factor authentication. \\
	{\bf Rationale:} Implementing authentication mechanisms, such as multi-factor authentication, can help to protect against unauthorized access and enhance the overall security of the system.
\end{enumerate}
% End SubSubSection
% End SubSection

\subsection{Cultural and Political Requirements}
\label{sub:cultural_and_political_requirements}
% Begin SubSection

\subsubsection{Cultural Requirements}
\label{ssub:cultural_requirements}
% Begin SubSubSection
\begin{enumerate}[{CP-C}1. ]
	\item All dashboard text and alerts shall be written in plain language and avoid technical terminology or unexplained abbreviations. \\
    {\bf Rationale:} Users may not understand technical terminology. Simple wording reduces confusion and misinterpretation.
\end{enumerate}

\begin{enumerate}[{CP-C}2. ]
	\item All dashboard content and alerts shall be provided in English for the first version of the system. \\
    {\bf Rationale:} Limiting to one language in the initial release reduces implementation complexity while ensuring consistent communication.
\end{enumerate}

\begin{enumerate}[{CP-C}3. ]
	\item Alerts shall avoid culturally-specific idioms or metaphors, and shall use neutral, literal wording. \\
    {\bf Rationale:} Idioms and metaphors don’t translate consistently across cultures. Neutral wording improves clarity and reduces misinterpretation.
\end{enumerate}
% End SubSubSection

\subsubsection{Political Requirements}
\label{ssub:political_requirements}
% Begin SubSubSection
N/A
% End SubSubSection

% End SubSection

\subsection{Legal Requirements}
\label{sub:legal_requirements}
% Begin SubSection

\subsubsection{Compliance Requirements}
\label{ssub:compliance_requirements}
% Begin SubSubSection
\begin{enumerate}[{LR-COMP}1. ]
	\item Personal information handled by the system must be protected against loss, theft, and unauthorized access, disclosure, copying, use, or modification [7]. \\
    {\bf Rationale:} This aligns with PIPEDA Principle 7 (Safeguards), which requires appropriate protection of personal information.
\end{enumerate}

\begin{enumerate}[{LR-COMP}2. ]
	\item The system must obtain the user’s knowledge and consent before collecting, using, or disclosing their personal information [8]. \\
    {\bf Rationale:} This aligns with PIPEDA Principle 3 (Consent).
\end{enumerate}

\begin{enumerate}[{LR-COMP}3. ]
	\item The system must limit collection of personal information to only what is necessary for the stated purpose(s) [9]. \\
    {\bf Rationale:} This aligns with PIPEDA Principle 4 (Limiting Collection).
\end{enumerate}
% End SubSubSection

\subsubsection{Standards Requirements}
\label{ssub:standards_requirements}
% Begin SubSubSection
\begin{enumerate}[{LR-STD}1. ]
	\item The system shall not use notifications for promotional or advertisement purposes; notifications shall only communicate system or environmental monitoring events. \\
    {\bf Rationale:} Notifications should be relevant and not used for unsolicited ads.
\end{enumerate}

\begin{enumerate}[{LR-STD}2. ]
	\item Authenticated sessions should automatically expire after a period of inactivity, requiring the user to log in again to access operator functions. \\
    {\bf Rationale:} This is a standard for app quality in terms of security and access control.
\end{enumerate}

\begin{enumerate}[{LR-STD}3. ]
	\item Any personal or sensitive data shall not be written to system logs or application-specific logs. \\
    {\bf Rationale:} This reduces the risk of sensitive data leakage through logs.
\end{enumerate}
% End SubSubSection

% End SubSection
}
