\newcommand{\product}{\section{Overall Product Description}
\label{sec:overall_description}

\subsection{Product Perspective}
\label{sub:product_perspective}
% Begin SubSection


\indent The Smart City Environmental Monitoring \& Alert System ({\bf SCEMAS}) is a cloud-native {\bf IoT} software platform. Unlike systems such as SimpliCity, which focus on supporting city services, internal business process, and citizen engagement strategies, {\bf SCEMAS} specifically collects, validates, stores, and analyzes environmental teleemtry to support real-time urban environmental monitoring and decision making.

\vspace{0.5em}

{\bf SCEMAS} is self-contained but interacts with multiple external systems. A network of {\bf IoT} sensors collect environmental data and communicate them to {\bf SCEMAS}. The system validates this incoming data, performs real-time aggregation, and calculates metrics to provide actionable insights into environmental conditions. Based on alert rules defined by administrators, {\bf SCEMAS} triggers alerts upon detecting rule violations, logs all events, and notifies any subscribed external systems.

\vspace{0.5em}

The system provides tailored interfaces for city operators and the public. City operators access a secure dashboard with real-time environmental metrics, system health, and active alerts. Public users access aggregated, non-sensitive environmental data through a simplified dashboard.

\vspace{0.5em}

The system will interact with multiple external systems. {\bf SCEMAS} will notify any subscribed external systems to notify them of any alert events and will also interface with various {\bf IoT} sensors to collect environmental data. 


\begin{figure}[h!]
    \centering
    \includegraphics[width=0.7\linewidth]{BlockDiagram.drawio.png}
    \caption{System Diagram}
\end{figure}

% End SubSection

\subsection{Product Functions}
\label{sub:product_functions}
% Begin SubSection

There will be 4 main modules in this product: the Account Service, Authentication Service, Telemetry Service, and Alert Service. The major functions of the software in the Account Service include creating accounts, updating account details, and logging in and out. Within the Authentication Service, the system handles assigning roles to users and verifying permissions to control access to protected resources. The Telemetry Service is responsible for receiving sensor data, validating its format and plausibility, and storing it for further processing. Finally, the Alert Service provides functionality for creating and updating alert rules, evaluating incoming telemetry against these rules, generating alerts when violations occur, viewing alert history, and notifying external systems.

\hspace*{-1cm}
\begin{table}[H]
\centering
\renewcommand{\arraystretch}{1.0}

\setlength{\tabcolsep}{1pt}
\setlength{\itemsep}{0pt}
\setlength{\parskip}{0pt}
\setlength{\parsep}{0pt}
\setlength{\topsep}{0pt}
\setlength{\partopsep}{0pt}

\newenvironment{compactitem}{
  \begin{itemize}
    \setlength{\itemsep}{0pt}
    \setlength{\parskip}{0pt}
    \setlength{\parsep}{0pt}
    \setlength{\topsep}{0pt}
    \setlength{\partopsep}{0pt}
}{
  \end{itemize}
}

\begin{tabularx}{\textwidth}{|>{\raggedright\arraybackslash}p{4cm}|X|}
\hline
\textbf{Module} & \textbf{Functions} \\
\hline

Account Service &
\begin{compactitem}
    \item \textbf{Create Account}
    \begin{compactitem}
        \item Allows a user to create an account
    \end{compactitem}
    \item \textbf{Update Account}
    \begin{compactitem}
        \item Allows users to update account details
    \end{compactitem}
    \item \textbf{Login / Logout}
    \begin{compactitem}
        \item Authenticates users and manages sessions
    \end{compactitem}
\end{compactitem}
\\
\hline

Authentication Service &
\begin{compactitem}
	\item \textbf{Assign Roles}
	\begin{compactitem}
		\item Assigns roles to users based on their account type
	\end{compactitem}
	\item \textbf{Verify Permissions}
	\begin{compactitem}
		\item Verifies user permissions for allowing access to protected resources
	\end{compactitem}
\end{compactitem}
\\
\hline

Telemetry Service &
\begin{compactitem}
    \item \textbf{Receive Telemetry Data}
    \begin{compactitem}
        \item Accepts sensor data via MQTT
    \end{compactitem}
	\item \textbf{Validate Telemetry Data}
    \begin{compactitem}
        \item Validate each incoming message for correct format, adherence to schema, and plausability
    \end{compactitem}
	\item \textbf{Store Telemetry Data}
    \begin{compactitem}
        \item Store validated telemetry data for further processing
    \end{compactitem}
\end{compactitem}
\\
\hline

Alert Service &
\begin{compactitem}
    \item \textbf{Create Alert Rule}
    \begin{compactitem}
        \item Allows operators to define new alert rules
    \end{compactitem}
    \item \textbf{Update Alert Rule}
    \begin{compactitem}
        \item Allows operators to update existing alert rules
    \end{compactitem}
	\item \textbf{Evaluate Telemetry Against Rules}
	\begin{compactitem}
        \item Continuously checks incoming data for violations
    \end{compactitem}
	\item \textbf{Create Alert}
    \begin{compactitem}
        \item Generate an alert when a rule is violated
    \end{compactitem}
    \item \textbf{Resolve Alert}
    \begin{compactitem}
        \item Allows operators to mark an alert as resolved
    \end{compactitem}
	\item \textbf{View Alert History}
	\begin{compactitem}
        \item Displays past and currently active alerts
    \end{compactitem}
	\item \textbf{Notify external systems}
	\begin{compactitem}
        \item Send alerts to subscribed systems
    \end{compactitem}
\end{compactitem}
\\
\hline

\end{tabularx}
\end{table}
\hspace*{-1cm}

\begin{figure}[h!]
    \centering
    \includegraphics[width=1.1\linewidth]{StateDiagram.drawio.png}
    \caption{State Diagram}
\end{figure}
% End SubSection

\subsection{User Characteristics}
\label{sub:user_characteristics}
% Begin SubSection
The {\bf SCEMAS} system is designed for two main types of users: the general public and city operators. The expected skills and knowledge for each are as follows:

\begin{enumerate}
    \item \textbf{Education Level:}
    \begin{itemize}
        \item \textbf{General Public:} Basic reading and writing skills
        \begin{itemize}
            \item Users should be able to understand simple environmental information, like air quality or temperature alerts.
        \end{itemize}
        \item \textbf{City Operators:} College-level or equivalent technical education
        \begin{itemize}
            \item Operators should have a good understanding of environmental science and/or urban planning concepts to interpret the data effectively.
        \end{itemize}
    \end{itemize}
    
    \item \textbf{Experience:}
    \begin{itemize}
        \item \textbf{General Public:} Any experience
        \begin{itemize}
            \item The public interface is easy to use, so even first-time users should be able to check environmental info without problems.
        \end{itemize}
        \item \textbf{City Operators:} Moderate to advanced experience
        \begin{itemize}
            \item Operators should have some experience with monitoring systems, dashboards, and interpreting environmental telemetry to make informed decisions.
        \end{itemize}
    \end{itemize}
    
    \item \textbf{Technical Skills:}
    \begin{itemize}
        \item \textbf{General Public:} Basic digital literacy
        \begin{itemize}
            \item Users should be able to use a web browser or app to view information and alerts.
        \end{itemize}
        \item \textbf{City Operators:} Comfortable using computer systems and dashboards
        \begin{itemize}
            \item Operators should be able to look at data, set up alerts, and understand what the numbers mean for the city.
        \end{itemize}
    \end{itemize}
\end{enumerate}
% End SubSection

\subsection{Constraints}
\label{sub:constraints}
% Begin SubSection
\begin{enumerate}
	\item \textbf{Project Time:} The time allotted for the project affects achievable scope, implementation depth and complexity.
    \item \textbf{Team Resources:} The team size and available effort limits implementation. The system must be buildable and testable by the assigned team size and skill set within the term.
    \item \textbf{Budget:} There is no budget for the project, so only free/open-source tools and free-tier services may be used.
\end{enumerate}
% End SubSection

\subsection{Assumptions and Dependencies}
\label{sub:assumptions_and_dependencies}
% Begin SubSection
\begin{enumerate}
    \item Sensors can send data continuously without requiring manual operator input.
    \item An alert is created only when at least one rule is violated.
    \item Operators/admins are responsible for creating/updating alert rules.
    \item The system depends on an Authentication/Identity service to validate credentials and assign roles.
    \item The system depends on a database to store telemetry, account info, and alert history.
    \item The system depends on network connectivity between sensors and the Telemetry Service for near real-time monitoring.
    \item Public dashboard content is derived from alerts/metrics and does not expose sensitive information.
    \item Sending alerts to subscribed systems depends on external APIs being reachable and stable.

\end{enumerate}
% End SubSection

\subsection{Apportioning of Requirements}
\label{sub:apportioning_of_requirements}
% Begin SubSection
\begin{enumerate}
    \item \textbf{Public dashboard enhancements:} Future versions may add zone/region filtering, richer visualizations, and user subscription preferences.
	\item \textbf{Advanced analytics and prediction:} Future versions may add trend analysis, anomaly detection, and early warning predictions.
    \item \textbf{Further notification capabilities:} Future versions may add SMS/push notifications.
    \item \textbf{Further language capabilities:} Future versions may add additional languages.
\end{enumerate}
% End SubSection
}