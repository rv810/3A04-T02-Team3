\newcommand{\product}{\section{Overall Product Description}
\label{sec:overall_description}

\subsection{Product Perspective}
\label{sub:product_perspective}
% Begin SubSection
\begin{itemize}
	\item Put the product into perspective with other related products, i.e., context
	\item If the product is independent and totally self-contained, it should be stated here
	\item If the SRS defines a product that is a component of a larger system, then this subsection should relate the requirements of that larger system to the functionality of the software being developed. Identify interfaces between that larger system and the software to be developed.
	\item A block diagram showing the major components of the larger system, interconnections, and external interfaces can be helpful
\end{itemize}

The Smart City Environmental Monitoring \& Alert System (SCEMAS) is a cloud-native IoT software platform. Unlike systems such as SimpliCity, which focus on supporting city services, internal business process, and citizen engagement strategies, SCEMAS specifically collects, validates, stores, and analyzes environmental teleemtry to support real-time uran environmental monitoring and decision making.


SCEMAS is self-contained but interacts with multiple external systems. A network of IoT sensors collect environmental data and communicate them to SCEMAS. The system validates this incoming data, performs real-time aggregation, and calculates metrics to provide actionable insights into environmental conditions. Based on alert rules defined by administrators, SCEMAS triggers alerts upon detecting rule violations, logs all events, and notifies any subscribed external systems.


The system provides tailored interfaces for city operators and the public. City operators access a secure dashboard with real-time environmental metrics, system health, and active alerts. Public users access aggregated, non-sensitive environmental data through a simplified dashboard.


The system will interact with multiple external systems. SCEMAS will notify any subscribed external systems to notify them of any alert events and will also interface with various IoT sensors to collect environmental data. 

% End SubSection

\subsection{Product Functions}
\label{sub:product_functions}
% Begin SubSection

\hspace*{-1cm}
\begin{table}[H]
\centering
\renewcommand{\arraystretch}{1.0}
\setlength{\tabcolsep}{3pt}
\begin{tabularx}{\textwidth}{|>{\raggedright\arraybackslash}p{4cm}|X|}
\hline
\textbf{Module} & \textbf{Functions} \\
\hline

Account Service &
\begin{itemize}
    \item \textbf{Create Account}
    \begin{itemize}
        \item Allows a user to create an account
    \end{itemize}
    \item \textbf{Update Account}
    \begin{itemize}
        \item Allows users to update account details
    \end{itemize}
    \item \textbf{Login / Logout}
    \begin{itemize}
        \item Authenticates users and manages sessions
    \end{itemize}
\end{itemize}
\\
\hline

Authentication Service &
\begin{itemize}
	\item \textbf{Assign Roles}
	\begin{itemize}
		\item Assigns roles to users based on their account type
	\end{itemize}
	\item \textbf{Verify Permissions}
	\begin{itemize}
		\item Verifies user permissions for allowing access to protected resources
	\end{itemize}
\end{itemize}
\\
\hline

Telemetry Service &
\begin{itemize}
    \item \textbf{Receive Telemetry Data}
    \begin{itemize}
        \item Accepts sensor data via MQTT
    \end{itemize}
	\item \textbf{Validate Telemetry Data}
    \begin{itemize}
        \item Validate each incoming message for correct format, adherence to schema, and plausability
    \end{itemize}
\end{itemize}
\\
\hline

Alert Rules Service &
\begin{itemize}
    \item \textbf{Create Alert Rule}
    \begin{itemize}
        \item Allows operators to define new alert rules
    \end{itemize}
    \item \textbf{Update Alert Rule}
    \begin{itemize}
        \item Allows operators to update existing alert rules
    \end{itemize}
	\item \textbf{Evaluate Telemetry Against Rules}
	\begin{itemize}
        \item Continuously checks incoming data for violations
    \end{itemize}
	\item \textbf{Create Alert}
    \begin{itemize}
        \item Generate an alert when a rule is violated
    \end{itemize}
	\item \textbf{View Alert History}
	\begin{itemize}
        \item Displays past and currently active alerts
    \end{itemize}
	\item \textbf{Notify external systems}
	\begin{itemize}
        \item Send alerts to subscribed systems
    \end{itemize}
\end{itemize}
\\
\hline

\end{tabularx}
\end{table}
\hspace*{-1cm}
% End SubSection

\subsection{User Characteristics}
\label{sub:user_characteristics}
% Begin SubSection
\begin{itemize}
	\item Describe those general characteristics of the intended users of the product including educational level, experience, and technical expertise 
	\item Since there will be many users, you may wish to divide into different user types or personas
%	\item Do not state specific requirements, but rather provide the reasons why certain specific requirements are later specified
\end{itemize}
% End SubSection

\subsection{Constraints}
\label{sub:constraints}
% Begin SubSection
\begin{itemize}
	\item Provide a general description of any constraints that will limit the developer's options
\end{itemize}
% End SubSection

\subsection{Assumptions and Dependencies}
\label{sub:assumptions_and_dependencies}
% Begin SubSection
\begin{itemize}
	\item List any assumptions you made in interpreting what the software being developed is aiming to achieve
	\item List any other assumptions you made that, if it fails to hold, could require you to change the requirements
	%\item List each of the factors that affect the requirements stated in the SRS
	%\item These factors are not design constraints on the software but are, rather, any changes to them that can affect the requirements in the SRS
	\begin{itemize}
		\item \textbf{Example}: An assumption may be that a specific operating system will be available on the hardware designated for the software product. If, in fact, the operating system is not available, the SRS would then have to change accordingly.
	\end{itemize}
\end{itemize}
% End SubSection

\subsection{Apportioning of Requirements}
\label{sub:apportioning_of_requirements}
% Begin SubSection
\begin{itemize}
	\item Identify requirements that may be delayed until future versions of the system
\end{itemize}
% End SubSection
}